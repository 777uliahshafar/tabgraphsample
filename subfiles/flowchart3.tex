\documentclass[../main.tex]{subfiles}

\begin{document}


\begin{figure}[hp]
\centering
\begin{tikzpicture}[node distance=2cm]
\node (ltr) [startstop] {Latar belakang};

\node (rum) [startstop, right of=ltr, xshift=2cm] {Perumusan masalah};

\node (tuj) [startstop, below of=rum, yshift=0.5cm] {Tujuan penelitian};

\node (pus) [startstop, below of=tuj, yshift=0.5cm] {Studi Pustaka};

\node (kaj) [startstop, below of=pus, text width=4cm, xshift= -4cm, yshift=-.2cm] {
	\textbf{Kajian Teori}\\ - Ruang publik waterfront\\ - Preferensi ruang\\ - Atribut ruang
};


\node (kaj2) [startstop, below of=pus, text width=3.5cm, xshift= 4cm, yshift=-.2cm] {
	\textbf{Gambaran Objek}\\  Kondisi ruang A dan B tepi Laut Senggol Parepare berdasarkan atribut ruang
};

\node (met) [startstop, below of=pus, yshift=-5cm, text width=7cm] {
	\textbf{Metode Peneltian}.\\
Penelitian ini bersifat eksploratif untuk menelusuri atribut yang ada yang terkait preferensi pengunjung terhadap tepi laut. Pendekatan \textit{Grounded theory} digunakan untuk memahami objek penelitian dengan pengguna (pengunjung).

	Sumber data: Observasi dan kuesioner survei dengan metodologi campuran.
};

\node (ana) [startstop, below of=met, yshift=-2cm, text width=8cm ] {
		\textbf{Analisis data}

         };

\node (tem) [startstop, below of=ana] {\textbf{Temuan Penelitian}};

\node (kes) [startstop, below of=tem, yshift=.6cm, text width=6cm] {Kesimpulan, keterbatasan dan rekomendasi};

\draw [arrow] (ltr) -- (rum);
\draw [arrow] (rum) -- (tuj);
\draw [arrow] (tuj) -- (pus);

\draw [arrow] (pus) -| (kaj);
\draw [arrow] (pus) -| (kaj2);

\draw [doublearrow] (kaj) -- (kaj2);

\draw [arrow] (kaj) |- (met);

\draw [arrow] (kaj2) |- (met);

\draw [arrow] (met) -- (ana);
\draw [arrow] (ana) -- (tem);

\draw [arrow] (tem) -- (kes);

\end{tikzpicture}
\caption{Kerangka Pemikiran}
\label{fig:pik}
\end{figure}

\begin{comment}
\begin{figure}[htbp]
\centering
\begin{tikzpicture}[node distance=2cm]


    \node (fit) [startstop, text width=\textwidth] {\textbf{Permalahan penelitian}\\ 1. Bagaimana preferensi pengunjung terhadap ruang di ruang­ruang tepi laut?  \\ 2. Bagaimana hubungan keseluruhan elemen dan preferensi ruang? };

	\node (soc) [startstop, below of=fit, text width=\textwidth, yshift=-1cm] {\textbf{Tujuan penelitian}\\ 1.Untuk mengetahui jenis ruang yang paling dicenderungi di kawasan tepi laut.
    \\ 2. Untuk menganalisis pola preferensi pengunjung terhadap keseluruhan elemen ruang. };

	\node (met) [startstop, below of=soc, text width=\textwidth, yshift=-1cm] {\textbf{Metode penelitian}\\ Penelitian eksploratif yang menggunakan kuesioner survei \textit{mixed-methodology} dan pendekatan \textit{grounded theory}. Teori dari hasil pendekatan tersebut tergabung dalam eksplorasi.};

    \node (pro) [startstop, below of=met, text width=\textwidth, yshift=-1cm] {\textbf{Proposisi}\\ Penyusunan proposisi penelitian berdasarkan tinjauan teory terkait yang disesuaikan dengan objek dan tujuan penelitian untuk dijadikan pedoman dalam pengumpulan dan analisis data.
    };

    \node (ker) [startstop, below of=pro, text width=8cm, yshift=-.5cm] {\textbf{Kerangka indikator penelitian}\\ Waterfront    };

    \node (keb) [startstop, below of=ker, text width=6cm, yshift=.5cm] {Kebutuhan masyarakat};

    \node (rup) [startstop, below of=keb, xshift=-3cm, text width= 3cm] {Ruang publik};

    \node (att) [startstop, below of=rup, text width= 4cm] {\textbf{Elemen}\\\small(laut, pohon, kedai, toilet, dll)};

    \node (jun) [startstop, below of=keb, xshift=3cm, text width= 3cm] {Pengunjung};

    \node (kar) [startstop, below of=jun, text width= 5cm] {\textbf{Karakter Pengunjung}\\\small(umur,gender,ras dan status kegiatan ekonomi)};

    \node (pre) [startstop, below of=att, xshift=3cm, yshift=-1cm, text width= 4cm] {\bfseries Preferensi pengunjung terhadap ruang publik};




\draw [arrow] (fit) -- (soc);
\draw [arrow] (soc) -- (met);
\draw [arrow] (met) -- (pro);
\draw [arrow] (pro) -- (ker);
\draw [arrow] (keb) -- (rup);
\draw [arrow] (rup) -- (att);
\draw [arrow] (keb) -- (jun);
\draw [arrow] (jun) -- (kar);

\draw [latex'-latex'] (att) -- (kar);

\draw [arrow] (att) -- (pre);
\draw [arrow] (kar) -- (pre);



\end{tikzpicture}
\caption{Kerangka Konseptual}
\end{figure}
\end{comment}
%%


\begin{figure}[htbp]
\small
\centering
\begin{tikzpicture}[node distance=1cm]

    \node (fit) [startstop, text width=2.5cm, xshift=-6cm] {\textbf{Preferensi terhadap ruang}};

	\node (soc) [startstop, right of=fit, text width=2.5cm, xshift=7cm] {\textbf{Ruang publik tepi laut}};

	\node (rrt) [startstop, below of=fit, text width=2.5cm, yshift=-3.5cm] {\textbf{Pelaku}\\

            \begin{tabular}{@{}p{.9\textwidth}@{}}
    \tabitem Gender  \\
    \tabitem Usia \\
    \tabitem Suku \\
    \tabitem Pendidikan \\
\end{tabular}};

	\node (met) [startstop, right of=rrt, text width=2.5cm, xshift=2.5cm] {\textbf{Aspek ruang}

            \begin{tabular}{@{}p{.9\textwidth}@{}}
    \tabitem Aksesibilitas  \\
    \tabitem Keamanan \\
    \tabitem Estetika \\
    \tabitem Fasilitas \\
\end{tabular}};


	\node (ele) [startstop, right of=rrt, text width=4cm, xshift=7cm] {\textbf{Elemen ruang}

            \begin{tabular}{@{}p{.9\textwidth}@{}}
    \tabitem Jumlah pohon  \\
    \tabitem Bentuk pohon \\
    \tabitem Lebar jalan \\
    \tabitem Permukaan jalan \\
    \tabitem Warna bunga \\
    \tabitem Jenis kursi \\
    \tabitem Tingkat cahaya \\
    \tabitem Orientasi elemen \\
    \tabitem Tempat wisata air \\
    \tabitem Bgnan penunjang \\
    \tabitem Elemen air \\
\end{tabular}};


 \draw[thick,dashed] (-7.8, -20) rectangle (4.4, -7.8);



    \node (elm) [below of=met, yshift= -8.2cm, text width= .8\textwidth] {\centering\textbf{Variabel Penelitian}

            \begin{longtable}{@{}p{.33\textwidth}@{}p{.33\textwidth}@{}p{.43\textwidth}}
Aksesibilitas & Lebar jalan & Jalan pedestrian yang aksesibel memperhatikan lebar jalan \\
 & Permukaan jalan & Permukaan terlibat dalam laju dan pilihan rute pejalan kaki\\
Keamanan & Bentuk pohon & Bentuk pohon tertentu memberikan rasa aman\\
 & Pencahayaan jalan & Pencahayaan mendorong penggunaan ruang publik 24 jam dan meningkatkan rasa aman individu\\
Estetika & Jumlah pohon& Jumlah pohon tertentu meningkatkan kenikmatan estetika\\
 & Bunga yang berwarna & Bunga pada desain lanskap memberikan ketertarikan visual\\
 & Elemen air & Elemen air merupakan atraksi yang menarik pada ruang publik tepi laut\\


\end{longtable}
    };

\draw [white,-{Stealth}] (fit) -- (soc) node(xu) [midway] {};
\draw [-{Stealth[length=3mm,width=2mm]}] (fit) -- (-2,0);
\draw [{Stealth[length=3mm,width=2mm]}-] (-2,0) -- (soc);
\draw [arrow] (fit) -- (rrt);

\draw [arrow] (soc) -- (ele);

\draw [white,-{Stealth}] (met) -- (ele) node(yu) [midway] {};
\draw [{Rays[n=6,scale=4]}-{Stealth[scale=2]}] (yu) -- (-.57,-7.6);

\draw [arrow] (rrt) -- (met);

\draw [arrow] (met) -- (ele);



\end{tikzpicture}
\end{figure}

\begin{figure}[htbp]
\small
\centering
\begin{tikzpicture}[node distance=1cm]

 \draw[thick,dashed] (-5.8, 12) rectangle (6.5, 26);
    \node (elm) [below of=met, yshift=25cm, xshift=1.98cm, text width= .8\textwidth] {\centering\textbf{Variabel Penelitian}

            \begin{longtable}{@{}p{.33\textwidth}@{}p{.33\textwidth}@{}p{.43\textwidth}}

Fasilitas & Bentuk pohon& Bentuk pohon tertentu memfasilitasi aktivitas fisik dan interaksi sosial\\
 & Jenis kursi & Kursi yang beranekaragam akan menciptakan beragam penggunaan. Beberapa diantara seperti bersosialisasi, melihat-lihat dan bersantai \\
 & Lebar jalan & Lebar jalan tertentu dapat memfasilitasi mobilitas, interaksi sosial, dan kenyamanan\\
 & Orientasi elemen & Orientasi elemen memfasilitasi pengunjung untuk melihat panorama dan pemandangan.  \\
 & Tempat wisata laut & Tempat ini merupakan area dimana orang dapat berinteraksi dengan air\\
 & Bangunan penunjang & Bangunan penunjang merupakan bangunan yang dapat mengakomodasi kegiatan-kegiatan pengunjung. \\
\end{longtable}
    };

  \path
    (3.8, 12.5) node(rec)[below left] {Variabel ini sebagai alat analisis untuk dilapangan}
  ;


    \node (jun) [startstop, below of=rec, yshift= -1cm, text width= 4.5cm] {\textbf{Tepi laut Senggol}\\ Ruang A dan Ruang B};

    \node (tem) [startstop, below of=jun, yshift= -.8cm, text width= 4.5cm] {Menghasilkan temuan preferensi pada ruang publik tepi laut};

\draw [arrow] (rec) -- (jun);
\draw [arrow] (elm) -- (rec);
\draw [arrow] (jun) -- (tem);

\end{tikzpicture}
\caption{Kerangka Konseptual}
\label{fig:conc2}
\end{figure}




\begin{figure}[htbp]
\small
\centering
\begin{tikzpicture}[node distance=1cm]

    \node (fit) [startstop, text width=.83\textwidth] {\textbf{Permasalahan penelitian}\\ 1. Ruang apa yang dipilih pengunjung untuk dimanfaatkan sebagai ruang publik?  \\ 2. Mengapa orang memiliki preferensi pada ruang tersebut? \\ 3. Apa saja elemen-elemen yang ada pada ruang publik sehingga pengunjung lebih memilih ruang tersebut? };

	\node (soc) [startstop, below of=fit, text width=.83\textwidth, yshift=-2.3cm] {\textbf{Tujuan penelitian}\\
    1.  Untuk menganalisis kecenderungan pengunjung dalam memilih ruang dikawasan tepi laut Senggol.\\
    2.  Untuk menganalisis latar belakang yang mendasari pemilihan tersebut.\\
    3.  Untuk menganalisis apa saja elemen ruang publik yang disukai.\\

        };

	\node (met) [startstop, below of=soc, text width=.83\textwidth, yshift=-1.6cm] {\textbf{Metode penelitian}\\ Pendekatan penilitian menggunakan metode kualitatif dan kuatitatif};

    \node (pro) [startstop, below of=met, text width=.83\textwidth, yshift=-1cm] {\textbf{Proposisi}\\ Penyusunan proposisi penelitian berdasarkan tinjauan teori terkait yang disesuaikan dengan objek dan tujuan penelitian untuk dijadikan pedoman dalam pengumpulan dan analisis data.
    };

    \node (ker) [startstop, below of=pro,  yshift=-.8cm, text width=6cm] {\textbf{Kerangka indikator penelitian}    };

    \node (ask) [startstop, below of=ker, xshift=-2.4cm, yshift=-.6cm, text width= 6.2cm] {\textbf{ASPEK RUANG}\\

\begin{tabular}{p{.4\textwidth}@{}p{.4\textwidth}}
    \tabitem Aksesibilitas & \tabitem Estetika \\
    \tabitem Keamanan & \tabitem Fasilitas\\
\end{tabular}};

    \node (elm) [startstop, below of=ask, yshift= -3cm, text width= 6.2cm] {\textbf{ELEMEN RUANG}

            \begin{tabular}{@{}p{.4\textwidth}@{}p{.4\textwidth}}
    \tabitem Jumlah pohon & \tabitem Warna bunga \\
    \tabitem Bentuk pohon & \tabitem Jenis kursi\\
    \tabitem Lebar jalan & \tabitem Pencahayaan jalan\\
    \tabitem Permukaan jalan & \\
\end{tabular}};


    \node (jun) [startstop, below of=ker, xshift=3.6cm, yshift= -2.4cm, text width= 4.5cm] {Menghasilkan temuan preferensi pada ruang publik tepi laut};

\draw [arrow] (fit) -- (soc);
\draw [arrow] (soc) -- (met);
\draw [arrow] (met) -- (pro);
\draw [arrow] (pro) -- (ker);

\draw [doublearrow] (ask) -- (elm);

\draw [arrow] (ask) -| (jun);
\draw [arrow] (elm) -| (jun);

\end{tikzpicture}
\caption{Kerangka Konseptual}
\label{fig:conc}
\end{figure}



\begin{figure}[htbp]
\small
\centering
\begin{tikzpicture}[node distance=1cm]

    \node (fit) [startstop, text width=.83\textwidth] {\textbf{Ruang terbuka publik}\\ adalah ruang yang berperan untuk memberi alur pergerakan yang baik, bertindak sebagai tempat berkumpul dan wadah kegiatan.};

	\node (soc) [startstop, below of=fit, text width=.83\textwidth, yshift=-1.4cm] {\textbf{Tepi laut}\\ adalah kawasan dinamis yang berbatasan dengan air yang memiliki kontak fisik dan visual dengan laut, sungai, danau, dan lain-lain.
        };

	\node (rrt) [startstop, below of=soc, text width=.83\textwidth, yshift=-1.4cm] {\textbf{Jenis ruang tepi laut}\\

            \begin{tabular}{@{}p{.4\textwidth}@{}p{.4\textwidth}}
    \tabitem Ruang rekreasi & \tabitem Ruang komersial \\
    \tabitem Ruang Alamiah & \tabitem Ruang Perumahan\\
    \tabitem Ruang Kerja &  \\
\end{tabular}};
	\node (met) [startstop, below of=rrt, text width=.83\textwidth, yshift=-1.4cm] {\textbf{Preferensi ruang}\\ adalah kecenderungan dalam memilih sesutu yang lebih disukai daripada yang lain};

 \draw[thick,dashed] (-6, -15.5) rectangle (6, -8.5);
  \path
    (5, -14.8) node(rec)[below left] {Proposisi ini sebagai alat analisis untuk dilapangan}
  ;

    \node (ask) [startstop, below of=met, xshift=-3cm, yshift=-3cm, text width= 5.2cm] {\textbf{Aspek ruang yang mempengaruhi preferensi}\\

\begin{tabular}{p{.4\textwidth}@{}p{.4\textwidth}}
    \tabitem Aksesibilitas & \tabitem Estetika \\
    \tabitem Keamanan & \tabitem Fasilitas\\
\end{tabular}};

    \node (elm) [startstop, below of=met, xshift=3cm, yshift= -3.3cm, text width= 5.2cm] {\textbf{Elemen ruang yang mempengaruhi preferensi berdasarkan aspek}

            \begin{tabular}{@{}p{.4\textwidth}@{}p{.4\textwidth}}
    \tabitem Jumlah pohon & \tabitem Warna bunga \\
    \tabitem Bentuk pohon & \tabitem Jenis kursi\\
    \tabitem Lebar jalan & \tabitem Pencahayaan jalan\\
    \tabitem Permukaan jalan & \\
\end{tabular}};


    \node (jun) [startstop, below of=ask, xshift=3.6cm, yshift= -4.2cm, text width= 4.5cm] {\textbf{Tepi laut Senggol}\\ Ruang A dan Ruang B};

    \node (tem) [startstop, below of=jun, yshift= -1cm, text width= 4.5cm] {Menghasilkan temuan preferensi pada ruang publik tepi laut};
\draw [arrow] (fit) -- (soc);
\draw [arrow] (soc) -- (rrt);

\draw [arrow] (rrt) -- (met);

\draw [arrow] (met) -- (ask);
\draw [arrow] (met) -- (elm);

\draw [doublearrow] (ask) -- (elm);

\draw [arrow] (ask) -- (rec);
\draw [arrow] (elm) -- (rec);
\draw [arrow] (rec) -- (jun);
\draw [arrow] (jun) -- (tem);

\end{tikzpicture}
\caption{Kerangka Konseptual}
\label{fig:conc}
\end{figure}


\begin{figure}[hp]
\centering
\begin{tikzpicture}[node distance=1.4cm]
\node (ltr) [startstop] {Latar belakang};

\node (rum) [startstop, right of=ltr, xshift=2.8cm] {Perumusan masalah};

\node (tuj) [startstop, below of=rum] {Tujuan penelitian};

\node (pus) [startstop, below of=tuj] {Studi Pustaka};

\node (kaj) [startstop, below of=pus, text width=4cm, xshift= -4cm, yshift=-1cm] {
	\textbf{Kajian Teori}\\ - Ruang publik waterfront \\ - Jenis Ruang\\ - Aspek dan Elemen ruang
};


\node (kaj2) [startstop, below of=pus, text width=3.5cm, xshift= 4cm, yshift=-1cm] {
	\textbf{Gambaran Objek}\\  Kondisi ruang A dan B tepi Laut Senggol Parepare.
};

\node (met) [startstop, below of=pus, yshift=-6cm, text width=7cm] {
	\textbf{Metode Peneltian}.\\
Pendekatan penelitian ini adalah kualitatif dan kuantitatif untuk menganalisis kecenderungan pengunjung dalam memilih ruang, mengetahui latar belakang yang mendasari pemilihan tersebut, dan menganalisis apa saja elemen ruang publik yang disukai.
% tidak perlu menggunakan istilah asing sperti grounded theory dll.
% 11:15
%Pendekatan \textit{Grounded theory} digunakan untuk memahami objek penelitian dengan pengguna (pengunjung).

	Sumber data: Observasi dan kuesioner survei dengan metodologi campuran.
};

\node (pro) [startstop, below of=met, yshift=-3.4cm, text width=8cm ] {
		\textbf{Proposisi}\\ Penyusunan proposisi penelitian berdasarkan tinjauan pustaka disesuaikan dengan objek dan tujuan penelitian untuk dijadikan pedoman dalam pengumpulan data dan analisis.
         };

\node (ana) [startstop, below of=pro, yshift=-1.4cm,  text width=8cm ] {
		\textbf{Analisis data}\\ Analisis menggunakan crosstab dan multivariate-biplot.
         };

\node (tem) [startstop, below of=ana, yshift=-.4cm] {\textbf{Temuan Penelitian}};

\node (kes) [startstop, below of=tem, text width=6cm] {Kesimpulan dan rekomendasi};

\draw [arrow] (ltr) -- (rum);
\draw [arrow] (rum) -- (tuj);
\draw [arrow] (tuj) -- (pus);

\draw [arrow] (pus) -| (kaj);
\draw [arrow] (pus) -| (kaj2);

\draw [doublearrow] (kaj) -- (kaj2);

\draw [arrow] (kaj) |- (met);

\draw [arrow] (kaj2) |- (met);

\draw [arrow] (met) -- (pro);
\draw [arrow] (pro) -- (ana);
\draw [arrow] (ana) -- (tem);

\draw [arrow] (tem) -- (kes);

\end{tikzpicture}
\caption{Kerangka Pemikiran}
\label{fig:pik}
\end{figure}




%\onlyinsubfile{\biblio}
\end{document}
