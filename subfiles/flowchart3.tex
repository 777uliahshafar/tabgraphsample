\documentclass[../main.tex]{subfiles}

\begin{document}


\begin{figure}[hp]
\centering
\begin{tikzpicture}[node distance=2cm]
\node (ltr) [startstop] {Latar belakang};

\node (rum) [startstop, right of=ltr, xshift=2cm] {Perumusan masalah};

\node (tuj) [startstop, below of=rum, yshift=0.5cm] {Tujuan penelitian};

\node (pus) [startstop, below of=tuj, yshift=0.5cm] {Studi Pustaka};

\node (kaj) [startstop, below of=pus, text width=4cm, xshift= -4cm, yshift=-.2cm] {
	\textbf{Kajian Teori}\\ - Ruang publik waterfront\\ - Preferensi ruang\\ - Atribut ruang
};


\node (kaj2) [startstop, below of=pus, text width=3.5cm, xshift= 4cm, yshift=-.2cm] {
	\textbf{Gambaran Objek}\\  Kondisi ruang A dan B tepi Laut Senggol Parepare berdasarkan atribut ruang
};

\node (met) [startstop, below of=pus, yshift=-5cm, text width=7cm] {
	\textbf{Metode Peneltian}.\\
Penelitian ini bersifat eksploratif untuk menelusuri atribut yang ada yang terkait preferensi pengunjung terhadap tepi laut. Pendekatan \textit{Grounded theory} digunakan untuk memahami objek penelitian dengan pengguna (pengunjung).

	Sumber data: Observasi dan kuesioner survei dengan metodologi campuran.
};

\node (ana) [startstop, below of=met, yshift=-2cm, text width=8cm ] {
		\textbf{Analisis data}

         };

\node (tem) [startstop, below of=ana] {\textbf{Temuan Penelitian}};

\node (kes) [startstop, below of=tem, yshift=.6cm, text width=6cm] {Kesimpulan, keterbatasan dan rekomendasi};

\draw [arrow] (ltr) -- (rum);
\draw [arrow] (rum) -- (tuj);
\draw [arrow] (tuj) -- (pus);

\draw [arrow] (pus) -| (kaj);
\draw [arrow] (pus) -| (kaj2);

\draw [doublearrow] (kaj) -- (kaj2);

\draw [arrow] (kaj) |- (met);

\draw [arrow] (kaj2) |- (met);

\draw [arrow] (met) -- (ana);
\draw [arrow] (ana) -- (tem);

\draw [arrow] (tem) -- (kes);

\end{tikzpicture}
\caption{Kerangka Pemikiran}
\label{fig:pik}
\end{figure}


\begin{figure}[htbp]
\centering
\begin{tikzpicture}[node distance=2cm]


    \node (fit) [startstop, text width=\textwidth] {\textbf{Permalahan penelitian}\\ 1. Bagaimana preferensi pengunjung terhadap ruang di ruang­ruang tepi laut?  \\ 2. Bagaimana hubungan keseluruhan elemen dan preferensi ruang? };

	\node (soc) [startstop, below of=fit, text width=\textwidth, yshift=-1cm] {\textbf{Tujuan penelitian}\\ 1.Untuk mengetahui jenis ruang yang paling dicenderungi di kawasan tepi laut.
    \\ 2. Untuk menganalisis pola preferensi pengunjung terhadap keseluruhan elemen ruang. };

	\node (met) [startstop, below of=soc, text width=\textwidth, yshift=-1cm] {\textbf{Metode penelitian}\\ Penelitian eksploratif yang menggunakan kuesioner survei \textit{mixed-methodology} dan pendekatan \textit{grounded theory}. Teori dari hasil pendekatan tersebut tergabung dalam eksplorasi.};

    \node (pro) [startstop, below of=met, text width=\textwidth, yshift=-1cm] {\textbf{Proposisi}\\ Penyusunan proposisi penelitian berdasarkan tinjauan teory terkait yang disesuaikan dengan objek dan tujuan penelitian untuk dijadikan pedoman dalam pengumpulan dan analisis data.
    };

    \node (ker) [startstop, below of=pro, text width=8cm, yshift=-.5cm] {\textbf{Kerangka indikator penelitian}\\ Waterfront    };

    \node (keb) [startstop, below of=ker, text width=6cm, yshift=.5cm] {Kebutuhan masyarakat};

    \node (rup) [startstop, below of=keb, xshift=-3cm, text width= 3cm] {Ruang publik};

    \node (att) [startstop, below of=rup, text width= 4cm] {\textbf{Elemen}\\\small(laut, pohon, kedai, toilet, dll)};

    \node (jun) [startstop, below of=keb, xshift=3cm, text width= 3cm] {Pengunjung};

    \node (kar) [startstop, below of=jun, text width= 5cm] {\textbf{Karakter Pengunjung}\\\small(umur,gender,ras dan status kegiatan ekonomi)};

    \node (pre) [startstop, below of=att, xshift=3cm, yshift=-1cm, text width= 4cm] {\bfseries Preferensi pengunjung terhadap ruang publik};




\draw [arrow] (fit) -- (soc);
\draw [arrow] (soc) -- (met);
\draw [arrow] (met) -- (pro);
\draw [arrow] (pro) -- (ker);
\draw [arrow] (keb) -- (rup);
\draw [arrow] (rup) -- (att);
\draw [arrow] (keb) -- (jun);
\draw [arrow] (jun) -- (kar);

\draw [latex'-latex'] (att) -- (kar);

\draw [arrow] (att) -- (pre);
\draw [arrow] (kar) -- (pre);



\end{tikzpicture}
\caption{Kerangka Konseptual}
\end{figure}

%\onlyinsubfile{\biblio}
\end{document}
