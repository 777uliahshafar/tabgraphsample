\documentclass[../main.tex]{subfiles}

\begin{document}

\begin{figure}[htbp]
\centering
\begin{tikzpicture}[node distance=2cm]

	\node (fit) [startstop, text width= 6cm] {\textbf{Elemen Ruang Publik}\\\small(Aksesibilitas, Estetika, Keamanan dan Fasilitas)};

	\node (pre) [startstop, right of=fit, text width= 5cm, xshift=4cm] {\textbf{Preferensi Pengunjung}\\ Kognisi, Afektif, Interpretatif, Evaluatif };

	\node (soc) [startstop, below of=fit, text width= 8cm, xshift= 3cm, yshift=-1cm ] {\textbf{Preferensi Ruang}};



\draw [arrow] (fit) -- (soc);
\draw [arrow] (pre) -- (soc);


\end{tikzpicture}
\caption{Kerangka Konseptual}
\end{figure}

%%

\begin{landscape}
\renewcommand{\arraystretch}{0.8}\setlength\tabcolsep{1pt}
\begin{longtable}{p{.33\textwidth}D{\%}{\%}{4.-1}*{2}{D{\%}{\%}{3.-1}}D{\%}{\%}{4.-1}*{3}{D{\%}{\%}{3.-1}}D{\%}{\%}{4.-1}*{2}{D{\%}{\%}{3.-1}}D{\%}{\%}{4.-1}*{3}{D{\%}{\%}{3.-1}}
    }
	\caption{Variabel Penelitian \footnotesize{samb.}}\\
	\label{tab:props}\\
\hline
\bfseries\diagbox[innerleftsep=10pt,innerrightsep=3pt,width=11em, height=3.3cm]{Elemen\\Ruang}{Pelaku} &

{\rotatebox[origin=c]{90}{\parbox[c]{3cm}{\textbf{Gender}}}} & {\rotatebox[origin=c]{90}{\parbox[c]{3.3cm}{Laki-laki}}} & {\rotatebox[origin=c]{90}{\parbox[c]{3.3cm}{Perempuan}}} &

{\rotatebox[origin=c]{90}{\parbox[c]{3cm}{\textbf{Usia}}}} &
{\rotatebox[origin=c]{90}{\parbox[c]{3.3cm}{Remaja}}}&
{\rotatebox[origin=c]{90}{\parbox[c]{3.3cm}{Dewasa}}}&
{\rotatebox[origin=c]{90}{\parbox[c]{3.3cm}{Manula}}}&


{\rotatebox[origin=c]{90}{\parbox[c]{3cm}{\textbf{Suku}}}}&
{\rotatebox[origin=c]{90}{\parbox[c]{3.3cm}{Bugis}}}&
{\rotatebox[origin=c]{90}{\parbox[c]{3.3cm}{Non-bugis}}}&

{\rotatebox[origin=c]{90}{\parbox[c]{3cm}{\textbf{Pendidikan}}}}&
{\rotatebox[origin=c]{90}{\parbox[c]{3.3cm}{< SMA/sederajat}}}&
{\rotatebox[origin=c]{90}{\parbox[c]{3.3cm}{SMA/sederajat}}}&
{\rotatebox[origin=c]{90}{\parbox[c]{3.3cm}{Perguruan tinggi}}}\\
\toprule
\textbf{Aspek Ruang} &&&&&&&&&&&&&&\\
\multirow[t]{2}{*}{Aksesibiltitas} &&31&16&&17&19&11&&33&15&&4&30&13\\
&&21\pc&11\pc&&11\pc&12\pc&7\pc&&22\pc&9\pc&&11\pc&20\pc&9\pc\\
\arrayrulecolor{black!30}\cmidrule(lr){2-15}

\multirow[t]{2}{*}{Keamanan}
&&7&0&&5&0&2&&6&1&&1&5&1\\
&&4\pc&0&&3\pc&0&1\pc&&4\pc&1\pc&&1\pc&3\pc&1\pc\\
\cmidrule(lr){2-15}

\multirow[t]{2}{*}{Estetika}
&&33&13&&15&22&9&&32&14&&7&17&22\\
&&22\pc&9\pc&&10\pc&15\pc&6\pc&&21\pc&9\pc&&2\pc&20\pc&12\pc\\
\cmidrule(lr){2-15}

\multirow[t]{2}{*}{Fasilitas} &&34&16&&17&19&14&&37&13&&3&29&18\\
&&22\pc&10\pc&&11\pc&12\pc&9\pc&&25\pc&8\pc&&2\pc&20\pc&12\pc\\
\cmidrule(lr){2-15}

\multirow[t]{2}{*}{Total} &&105&45&&54&60&36&&108&42&&15&81&54\\
&&70\pc&30\pc&&36\pc&40\pc&24\pc&&72\pc&28\pc&&10\pc&54\pc&36\pc\\
\cmidrule(lr){2-15}


\textbf{Elemen Ruang} &&&&&&&&&&&&&&\\

\multirow{2}{*}{Jumlah pohon} &&16&4&&6&6&8&&14&6&&2&9&9\\
&&6\pc&2\pc&&2\pc&2\pc&3\pc&&5\pc&2\pc&&1\pc&3\pc&3\pc\\
\cmidrule(lr){2-15}
\quad\tabitems \small{Sedikit pohon} &&&&&&&&&&&&&&\\
\quad\tabitems \small{Beberapa pohon} &&&&&&&&&&&&&&\\

\multirow{2}{*}{Bentuk pohon} &&26&14&&10&17&13&&26&14&&5&15&20\\
&&9\pc&5\pc&&4\pc&6\pc&5\pc&&10\pc&5\pc&&1\pc&5\pc&7\pc\\
\cmidrule(lr){2-15}
\quad\tabitems \small{Cukup rindang} &&&&&&&&&&&&&&\\
\quad\tabitems \small{Rindang} &&&&&&&&&&&&&&\\


\multirow{2}{*}{Lebar jalan} &&36&15&&22&17&12&&36&15&&7&24&20\\
&&14\pc&5\pc&&8\pc&7\pc&5\pc&&14\pc&5\pc&&2\pc&8\pc&7\pc\\
\cmidrule(lr){2-15}
\quad\tabitems \small{1-3m} &&&&&&&&&&&&&&\\
\quad\tabitems \small{3m} &&&&&&&&&&&&&&\\

\multirow{2}{*}{Permukaan jalan} &&24&13&&14&16&7&&27&10&&4&17&16\\
&&8\pc&5\pc&&5\pc&5\pc&3\pc&&10\pc&3\pc&&1\pc&7\pc&5\pc\\
\cmidrule(lr){2-15}
\quad\tabitems \small{Paving} &&&&&&&&&&&&&&\\
\quad\tabitems \small{Aspal} &&&&&&&&&&&&&&\\
\quad\tabitems \small{Tanah} &&&&&&&&&&&&&&\\


\multirow{2}{*}{Jenis kursi} &&9&2&&3&4&4&&7&4&&1&6&4\\
&&3\pc&&&1\pc&1\pc&1\pc&&2\pc&1\pc&&&2\pc&2\pc\\
\cmidrule(lr){2-15}
\quad\tabitems \small{Kursi bergerak} &&&&&&&&&&&&&&\\
\quad\tabitems \small{Kursi dinding} &&&&&&&&&&&&&&\\

\multirow{2}{*}{Pencahayaan jalan} &&34&19&&18&22&13&&40&13&&5&24&24\\
&&13\pc&7\pc&&7\pc&8\pc&4\pc&&15\pc&5\pc&&1\pc&9\pc&9\pc\\
\cmidrule(lr){2-15}
\quad\tabitems \small{Sedang} &&&&&&&&&&&&&&\\
\quad\tabitems \small{Tinggi} &&&&&&&&&&&&&&\\

\multirow{2}{*}{Orientasi elemen} &&36&19&&19&23&13&&39&16&&7&19&29\\
&&14\pc&7\pc&&7\pc&8\pc&5\pc&&14\pc&6\pc&&2\pc&7\pc&10\pc\\
\cmidrule(lr){2-15}
\quad\tabitems \small{Membelakangi laut} &&&&&&&&&&&&&&\\
\quad\tabitems \small{Menghadap laut} &&&&&&&&&&&&&&\\

\multirow[t]{2}{*}{Total} &&181&86&&92&105&70&&189&78&&31&114&122\\
&&68\pc&32\pc&&35\pc&39\pc&26\pc&&71\pc&29\pc&&11\pc&43\pc&46\pc\\

%\multirow{2}{*}{Warna bunga} &&&&&&&&&&&&&&\\
%&&&&&&&&&&&&&&\\
%\quad\tabitems \small{satu atau dua warna} &&&&&&&&&&&&&&\\
%\quad\tabitems \small{tiga atau lebih warna} &&&&&&&&&&&&&&\\

%\multirow{2}{*}{Tempat wisata air} &&&&&&&&&&&&&&\\
%&&&&&&&&&&&&&&\\
%\quad\tabitems \small{Tempat memancing} &&&&&&&&&&&&&&\\
%\quad\tabitems \small{Tempat berenang} &&&&&&&&&&&&&&\\

%\multirow{2}{*}{Bangunan penunjang} &&&&&&&&&&&&&&\\
%&&&&&&&&&&&&&&\\
%\quad\tabitems \small{Stan} &&&&&&&&&&&&&&\\
%\quad\tabitems \small{Kedai} &&&&&&&&&&&&&&\\

%\multirow{2}{*}{Elemen air} &&&&&&&&&&&&&&\\
%&&&&&&&&&&&&&&\\
%\quad\tabitems \small{Laut tenang} &&&&&&&&&&&&&&\\
%\quad\tabitems \small{Laut berombak} &&&&&&&&&&&&&&\\
\bottomrule
\end{longtable}
\end{landscape}

%%

\begin{table}[!t]
\caption{Table Caption}
\label{tab1}
\centering
{\setlength\tabcolsep{0.1pt}%
\begin{tabular}{ccccccccccccccc}
& $D_1$ & $D_2$ & $D_3$ & $D_4$ & $D_5$ & $D_6$ && & $D_1$ & $D_2$ & $D_3$ & $D_4$ & $D_5$ & $D_6$ \\
$D_1$ & & \cca{30} & \cca{37} & \cca{12} & \cca{20} & \cca{0} && $D_1$ && \cca{0} & \cca{0} & \cca{2} & \cca{2} & \cca{2} \\
$D_2$ & \cca{0} & & \cca{2} & \cca{0} & \cca{2} & \cca{0} && $D_2$ & \cca{0} &  & \cca{3} & \cca{0} & \cca{0} & \cca{2} \\
$D_3$ & \cca{7} & \cca{4} &  & \cca{3} & \cca{0} & \cca{4} && $D_3$ & \cca{0} & \cca{4} &  & \cca{4} & \cca{2} & \cca{0} \\
$D_4$ & \cca{3} & \cca{0} & \cca{7} &  & \cca{4} & \cca{0} && $D_4$ & \cca{0} & \cca{0} & \cca{5} &  & \cca{0} & \cca{0} \\
$D_5$ & \cca{3} & \cca{7} & \cca{7} & \cca{2} &  & \cca{4} && $D_5$ & \cca{2} & \cca{2} & \cca{0} & \cca{4} &  & \cca{4} \\
$D_6$ & \cca{2} & \cca{2} & \cca{7} & \cca{2} & \cca{6} &  && $D_6$ & \cca{2} & \cca{2} & \cca{3} & \cca{0} & \cca{4} & \\
\end{tabular}}
\end{table}

%%

\renewcommand{\arraystretch}{0.8}\setlength\tabcolsep{1.2pt}
\begin{longtable}{p{.33\textwidth}R{1.2em} R{1.2em}R{1.2em}R{1.2em}R{1.2em}R{1.2em}R{1.2em}R{1.2em}R{1.2em}R{1.2em}R{1.2em}R{1.2em}R{1.2em}R{1.2em}}
	\caption{Variabel Penelitian}\\
	\label{tab:props}\\
\hline
\rowcolor{White}
\bfseries\diagbox[innerleftsep=10pt,innerrightsep=15pt,width=10em, height=3.3cm]{Elemen\\Ruang}{Pelaku} &

{\rotatebox[origin=c]{90}{\parbox[c]{3cm}{\textbf{Gender}}}} & {\rotatebox[origin=c]{90}{\parbox[c]{3.3cm}{Laki-laki}}} & {\rotatebox[origin=c]{90}{\parbox[c]{3.3cm}{Perempuan}}} &

{\rotatebox[origin=c]{90}{\parbox[c]{3cm}{\textbf{Usia}}}} &
{\rotatebox[origin=c]{90}{\parbox[c]{3.3cm}{Remaja}}}&
{\rotatebox[origin=c]{90}{\parbox[c]{3.3cm}{Dewasa}}}&
{\rotatebox[origin=c]{90}{\parbox[c]{3.3cm}{Manula}}}&


{\rotatebox[origin=c]{90}{\parbox[c]{3cm}{\textbf{Suku}}}}&
{\rotatebox[origin=c]{90}{\parbox[c]{3.3cm}{Bugis}}}&
{\rotatebox[origin=c]{90}{\parbox[c]{3.3cm}{Non-bugis}}}&

{\rotatebox[origin=c]{90}{\parbox[c]{3cm}{\textbf{Pendidikan}}}}&
{\rotatebox[origin=c]{90}{\parbox[c]{3.3cm}{< SMA/sederajat}}}&
{\rotatebox[origin=c]{90}{\parbox[c]{3.3cm}{SMA/sederajat}}}&
{\rotatebox[origin=c]{90}{\parbox[c]{3.3cm}{Perguruan tinggi}}}\\
\toprule
\endfirsthead
\caption{Variabel Penelitian \footnotesize{(samb.)}}\\
\hline
\rowcolor{White}
\bfseries\diagbox[innerleftsep=10pt,innerrightsep=15pt,width=11em, height=3.3cm]{Elemen\\Ruang}{Pelaku} &

{\rotatebox[origin=c]{90}{\parbox[c]{3cm}{\textbf{Gender}}}} & {\rotatebox[origin=c]{90}{\parbox[c]{3.3cm}{Laki-laki}}} & {\rotatebox[origin=c]{90}{\parbox[c]{3.3cm}{Perempuan}}} &

{\rotatebox[origin=c]{90}{\parbox[c]{3cm}{\textbf{Usia}}}} &
{\rotatebox[origin=c]{90}{\parbox[c]{3.3cm}{Remaja}}}&
{\rotatebox[origin=c]{90}{\parbox[c]{3.3cm}{Dewasa}}}&
{\rotatebox[origin=c]{90}{\parbox[c]{3.3cm}{Manula}}}&


{\rotatebox[origin=c]{90}{\parbox[c]{3cm}{\textbf{Suku}}}}&
{\rotatebox[origin=c]{90}{\parbox[c]{3.3cm}{Bugis}}}&
{\rotatebox[origin=c]{90}{\parbox[c]{3.3cm}{Non-bugis}}}&

{\rotatebox[origin=c]{90}{\parbox[c]{3cm}{\textbf{Pendidikan}}}}&
{\rotatebox[origin=c]{90}{\parbox[c]{3.3cm}{< SMA/sederajat}}}&
{\rotatebox[origin=c]{90}{\parbox[c]{3.3cm}{SMA/sederajat}}}&
{\rotatebox[origin=c]{90}{\parbox[c]{3.3cm}{Perguruan tinggi}}}\\
\toprule
\endhead
\textbf{Aspek Ruang} &&&&&&&&&&&&&&\\
Aksesibiltitas &  & \cca{31} & \cca{16} &  & \cca{17} & \cca{19} & \cca{11} &  & \cca{33} & \cca{15} &  & \cca{4} & \cca{30} & \cca{13}\\
Keamanan &  & \cca{7} & \cca{0} &  & \cca{5} & \cca{0} & \cca{2} &  & \cca{6} & \cca{1} &  & \cca{1} & \cca{5} & \cca{1}\\
Estetika &  & \cca{33} & \cca{13} &  & \cca{15} & \cca{22} & \cca{9} &  & \cca{32} & \cca{14} &  & \cca{7} & \cca{17} & \cca{22} \\
Fasilitas &  & \cca{34} & \cca{16} &  & \cca{17} & \cca{19} & \cca{14} &  & \cca{37} & \cca{13} &  & \cca{3} & \cca{29} & \cca{18} \\

\textbf{Elemen Ruang} &&&&&&&&&&&&&&\\

Jumlah pohon &  & \cca{16} & \cca{4} &  & \cca{6} & \cca{6} & \cca{8} &  & \cca{14} & \cca{6} &  & \cca{2} & \cca{9} & \cca{9} \\
\quad\tabitems \small{Sedikit pohon} &&&&&&&&&&&&&&\\
\quad\tabitems \small{Beberapa pohon} &&&&&&&&&&&&&&\\

Bentuk pohon &  & \cca{26} & \cca{14} &  & \cca{10} & \cca{17} & \cca{13} &  & \cca{26} & \cca{14} &  & \cca{5} & \cca{15} & \cca{20} \\
\quad\tabitems \small{Cukup rindang} &&&&&&&&&&&&&&\\
\quad\tabitems \small{Rindang} &&&&&&&&&&&&&&\\

Lebar jalan &  & \cca{36} & \cca{15} &  & \cca{22} & \cca{17} & \cca{12} &  & \cca{36} & \cca{15} &  & \cca{7} & \cca{24} & \cca{20} \\
\quad\tabitems \small{1-3m} &&&&&&&&&&&&&&\\
\quad\tabitems \small{3m} &&&&&&&&&&&&&&\\

Permukaan jalan &  & \cca{24} & \cca{13} &  & \cca{14} & \cca{16} & \cca{7} &  & \cca{27} & \cca{10} &  & \cca{4} & \cca{17} & \cca{16} \\
\quad\tabitems \small{Paving} &&&&&&&&&&&&&&\\
\quad\tabitems \small{Aspal} &&&&&&&&&&&&&&\\
\quad\tabitems \small{Tanah} &&&&&&&&&&&&&&\\

Warna bunga/tanaman  &&\ccg&\ccg&&\ccg&\ccg&\ccg&&\ccg&\ccg&&\ccg&\ccg&\ccg\\
\quad\tabitems \small{satu atau dua warna} &&&&&&&&&&&&&&\\
\quad\tabitems \small{tiga atau lebih warna} &&&&&&&&&&&&&&\\

Jenis kursi &  & \cca{9} & \cca{2} &  & \cca{3} & \cca{4} & \cca{4} &  & \cca{7} & \cca{4} &  & \cca{1} & \cca{6} & \cca{4} \\
\quad\tabitems \small{Kursi bergerak} &&&&&&&&&&&&&&\\
\quad\tabitems \small{Kursi dinding} &&&&&&&&&&&&&&\\

Pencahayaan jalan &  & \cca{34} & \cca{19} &  & \cca{18} & \cca{22} & \cca{13} &  & \cca{40} & \cca{13} &  & \cca{5} & \cca{24} & \cca{24} \\
\quad\tabitems \small{Sedang} &&&&&&&&&&&&&&\\
\quad\tabitems \small{Tinggi} &&&&&&&&&&&&&&\\

Orientasi Elemen &  & \cca{36} & \cca{19} &  & \cca{19} & \cca{23} & \cca{13} &  & \cca{39} & \cca{16} &  & \cca{7} & \cca{19} & \cca{29} \\
\quad\tabitems \small{Membelakangi laut} &&&&&&&&&&&&&&\\
\quad\tabitems \small{Menghadap laut} &&&&&&&&&&&&&&\\

Tempat wisata air  &&\ccg&\ccg&&\ccg&\ccg&\ccg&&\ccg&\ccg&&\ccg&\ccg&\ccg\\
\quad\tabitems \small{Tempat memancing} &&&&&&&&&&&&&&\\
\quad\tabitems \small{Tempat berenang} &&&&&&&&&&&&&&\\

Bangunan penunjang  &&\ccg&\ccg&&\ccg&\ccg&\ccg&&\ccg&\ccg&&\ccg&\ccg&\ccg\\
\quad\tabitems \small{Stan} &&&&&&&&&&&&&&\\
\quad\tabitems \small{Kedai} &&&&&&&&&&&&&&\\

Elemen air  &&\ccg&\ccg&&\ccg&\ccg&\ccg&&\ccg&\ccg&&\ccg&\ccg&\ccg\\
\quad\tabitems \small{Laut tenang} &&&&&&&&&&&&&&\\
\quad\tabitems \small{Laut berombak} &&&&&&&&&&&&&&\\
\bottomrule
 \multicolumn{15}{l}{\rule{0pt}{1em} \parbox{.9\textwidth}{Catatan: \noindent\textcolor{red}{\rule{1em}{5pt}} : berpengaruh tinggi
\noindent\textcolor{yellow!60}{\rule{1em}{5pt}} : berpengaruh sedang\par
\hspace*{1em}\textcolor{green!60}{\rule{1em}{5pt}} : berpengaruh rendah \textcolor{dgray}{\rule{1em}{5pt}} : berpengaruh sangat rendah

 }}\\
\end{longtable}

%%

\renewcommand{\arraystretch}{0.8}\setlength\tabcolsep{1.2pt}
\begin{longtable}{p{.33\textwidth}p{1.2em} p{1.2em}p{1.2em}p{1.2em}p{1.2em}p{1.2em}p{1.2em}p{1.2em}p{1.2em}p{1.2em}p{1.2em}p{1.2em}p{1.2em}p{1.2em}}
	\caption{Variabel Penelitian}\\
	\label{tab:varr}\\
\hline
\bfseries\diagbox[innerleftsep=10pt,innerrightsep=3pt,width=11em, height=3.3cm]{Elemen\\Ruang}{Pelaku} &

{\rotatebox[origin=c]{90}{\parbox[c]{3cm}{\textbf{Gender}}}} & {\rotatebox[origin=c]{90}{\parbox[c]{3.3cm}{Laki-laki}}} & {\rotatebox[origin=c]{90}{\parbox[c]{3.3cm}{Perempuan}}} &

{\rotatebox[origin=c]{90}{\parbox[c]{3cm}{\textbf{Usia}}}} &
{\rotatebox[origin=c]{90}{\parbox[c]{3.3cm}{Remaja}}}&
{\rotatebox[origin=c]{90}{\parbox[c]{3.3cm}{Dewasa}}}&
{\rotatebox[origin=c]{90}{\parbox[c]{3.3cm}{Manula}}}&


{\rotatebox[origin=c]{90}{\parbox[c]{3cm}{\textbf{Suku}}}}&
{\rotatebox[origin=c]{90}{\parbox[c]{3.3cm}{Bugis}}}&
{\rotatebox[origin=c]{90}{\parbox[c]{3.3cm}{Non-bugis}}}&

{\rotatebox[origin=c]{90}{\parbox[c]{3cm}{\textbf{Pendidikan}}}}&
{\rotatebox[origin=c]{90}{\parbox[c]{3.3cm}{< SMA/sederajat}}}&
{\rotatebox[origin=c]{90}{\parbox[c]{3.3cm}{SMA/sederajat}}}&
{\rotatebox[origin=c]{90}{\parbox[c]{3.3cm}{Perguruan tinggi}}}\\
\toprule
\endfirsthead
\caption{Variabel Penelitian \footnotesize{(samb.)}}\\
\hline
\bfseries\diagbox[innerleftsep=10pt,innerrightsep=3pt,width=11em, height=3.3cm]{Elemen\\Ruang}{Pelaku} &

{\rotatebox[origin=c]{90}{\parbox[c]{3cm}{\textbf{Gender}}}} & {\rotatebox[origin=c]{90}{\parbox[c]{3.3cm}{Laki-laki}}} & {\rotatebox[origin=c]{90}{\parbox[c]{3.3cm}{Perempuan}}} &

{\rotatebox[origin=c]{90}{\parbox[c]{3cm}{\textbf{Usia}}}} &
{\rotatebox[origin=c]{90}{\parbox[c]{3.3cm}{Remaja}}}&
{\rotatebox[origin=c]{90}{\parbox[c]{3.3cm}{Dewasa}}}&
{\rotatebox[origin=c]{90}{\parbox[c]{3.3cm}{Manula}}}&


{\rotatebox[origin=c]{90}{\parbox[c]{3cm}{\textbf{Suku}}}}&
{\rotatebox[origin=c]{90}{\parbox[c]{3.3cm}{Bugis}}}&
{\rotatebox[origin=c]{90}{\parbox[c]{3.3cm}{Non-bugis}}}&

{\rotatebox[origin=c]{90}{\parbox[c]{3cm}{\textbf{Pendidikan}}}}&
{\rotatebox[origin=c]{90}{\parbox[c]{3.3cm}{< SMA/sederajat}}}&
{\rotatebox[origin=c]{90}{\parbox[c]{3.3cm}{SMA/sederajat}}}&
{\rotatebox[origin=c]{90}{\parbox[c]{3.3cm}{Perguruan tinggi}}}\\
\toprule
\endhead
\textbf{Aspek Ruang} &&&&&&&&&&&&&&\\
Aksesibiltitas &&&&&&&&&&&&&&\\
Keamanan &&&&&&&&&&&&&&\\
Estetika &&&&&&&&&&&&&&\\
Fasilitas &&&&&&&&&&&&&&\\


\textbf{Elemen Ruang} &&&&&&&&&&&&&&\\

\textit{Jumlah pohon}&&&&&&&&&&&&&&\\
\tabitems Sedikit pohon&&&&&&&&&&&&&&\\
\tabitems Beberapa pohon&&&&&&&&&&&&&&\\
\textit{Bentuk pohon}&&&&&&&&&&&&&&\\
\tabitems Cukup rindang&&&&&&&&&&&&&&\\
\tabitems Rindang&&&&&&&&&&&&&&\\
\textit{Lebar jalan}&&&&&&&&&&&&&&\\
\tabitems 1-3m&&&&&&&&&&&&&&\\
\tabitems $\leq$ 3m&&&&&&&&&&&&&&\\

\textit{Permukaan jalan}&&&&&&&&&&&&&&\\
\tabitems Paving&&&&&&&&&&&&&&\\
\tabitems Aspal&&&&&&&&&&&&&&\\
\tabitems Tanah&&&&&&&&&&&&&&\\

\textit{Warna bunga/tanaman}&&&&&&&&&&&&&&\\
\tabitems satu atau dua warna&&&&&&&&&&&&&&\\
\tabitems tiga atau lebih warna&&&&&&&&&&&&&&\\
\textit{Jenis kursi}&&&&&&&&&&&&&&\\
\tabitems Kursi bergerak&&&&&&&&&&&&&&\\
\tabitems Kursi dinding&&&&&&&&&&&&&&\\

\textit{Pencahayaan jalan}&&&&&&&&&&&&&&\\
\tabitems Sedang&&&&&&&&&&&&&&\\
\tabitems Tinggi&&&&&&&&&&&&&&\\

\textit{Orientasi Elemen}  &&&&&&&&&&&&&&\\
\tabitems Membelakangi laut &&&&&&&&&&&&&&\\
\tabitems Menghadap laut   &&&&&&&&&&&&&&\\

\textit{Tempat Wisata air}  &&&&&&&&&&&&&&\\
\tabitems Tempat memancing  &&&&&&&&&&&&&&\\
\tabitems Tempat berenang   &&&&&&&&&&&&&&\\

\textit{Bangunan penunjang}  &&&&&&&&&&&&&&\\
\tabitems Stan \textit{(booth container)}   &&&&&&&&&&&&&&\\
\tabitems Kedai   &&&&&&&&&&&&&&\\

\textit{Elemen air}  &&&&&&&&&&&&&&\\
\tabitems Laut yang tenang   &&&&&&&&&&&&&&\\
\tabitems Laut yang berombak &&&&&&&&&&&&&&\\
\bottomrule
\end{longtable}


\begin{comment}
\begin{figure}[htbp]
\small
\centering
\begin{tikzpicture}[node distance=1cm]

    \node (fit) [startstop, text width=.83\textwidth] {\textbf{Ruang terbuka publik}\\ adalah ruang yang berperan untuk memberi alur pergerakan yang baik, bertindak sebagai tempat berkumpul dan wadah kegiatan.};

	\node (soc) [startstop, below of=fit, text width=.83\textwidth, yshift=-1.2cm] {\textbf{Tepi laut}\\ adalah kawasan dinamis yang berbatasan dengan air yang memiliki kontak fisik dan visual dengan laut, sungai, danau, dan lain-lain.
        };

	\node (rrt) [startstop, below of=soc, text width=.83\textwidth, yshift=-1.4cm] {\textbf{Jenis ruang tepi laut}\\

            \begin{tabular}{@{}p{.4\textwidth}@{}p{.4\textwidth}}
    \tabitem Ruang rekreasi & \tabitem Ruang komersial \\
    \tabitem Ruang Alamiah & \tabitem Ruang Perumahan\\
    \tabitem Ruang Kerja &  \\
\end{tabular}};
	\node (met) [startstop, below of=rrt, text width=.83\textwidth, yshift=-1.4cm] {\textbf{Preferensi ruang}\\ adalah kecenderungan dalam memilih sesutu yang lebih disukai daripada yang lain};

 \draw[thick,dashed] (-6, -16.6) rectangle (6, -8.5);
  \path
    (4.48, -15.8) node(rec)[below left] {Variabel ini sebagai alat analisis untuk dilapangan}
  ;

    \node (elm) [below of=met, yshift= -4cm, text width= .8\textwidth] {\centering\textbf{Variabel Penelitian}

            \begin{tabular}{p{.33\textwidth}p{1.2em} p{1.2em}p{1.2em}p{1.2em}p{1.2em}p{1.2em}p{1.2em}p{1.2em}p{1.2em}p{1.2em}p{1.2em}p{1.2em}p{1.2em}p{1.2em}}
\footnotesize
\bfseries\diagbox[innerleftsep=10pt,innerrightsep=3pt,width=9em, height=3.3cm]{Elemen\\Ruang}{Pelaku} &

{\rotatebox[origin=c]{90}{\parbox[c]{3cm}{\textbf{Gender}}}} & {\rotatebox[origin=c]{90}{\parbox[c]{3.3cm}{Laki-laki}}} & {\rotatebox[origin=c]{90}{\parbox[c]{3.3cm}{Perempuan}}} &

{\rotatebox[origin=c]{90}{\parbox[c]{3cm}{\textbf{Usia}}}} &
{\rotatebox[origin=c]{90}{\parbox[c]{3.3cm}{Remaja}}}&
{\rotatebox[origin=c]{90}{\parbox[c]{3.3cm}{Dewasa}}}&
{\rotatebox[origin=c]{90}{\parbox[c]{3.3cm}{Manula}}}&


{\rotatebox[origin=c]{90}{\parbox[c]{3cm}{\textbf{Suku}}}}&
{\rotatebox[origin=c]{90}{\parbox[c]{3.3cm}{Bugis}}}&
{\rotatebox[origin=c]{90}{\parbox[c]{3.3cm}{Non-bugis}}}&

{\rotatebox[origin=c]{90}{\parbox[c]{3cm}{\textbf{Pendidikan}}}}&
{\rotatebox[origin=c]{90}{\parbox[c]{3.3cm}{< SMA/sederajat}}}&
{\rotatebox[origin=c]{90}{\parbox[c]{3.3cm}{SMA/sederajat}}}&
{\rotatebox[origin=c]{90}{\parbox[c]{3.3cm}{Perguruan tinggi}}}\\
\toprule
\textbf{Aspek Ruang} &&&&&&&&&&&&&&\\
Aksesibilitas &&&&&&&&&&&&&&\\
\textit{dst...} &&&&&&&&&&&&&&\\
\textbf{Elemen Ruang} &&&&&&&&&&&&&&\\
Jumlah pohon &&&&&&&&&&&&&&\\
\textit{dst...} &&&&&&&&&&&&&&\\
\end{tabular}};


    \node (jun) [startstop, below of=elm, yshift= -4.6cm, text width= 4.5cm] {\textbf{Tepi laut Senggol}\\ Ruang A dan Ruang B};

    \node (tem) [startstop, below of=jun, yshift= -.8cm, text width= 4.5cm] {Menghasilkan temuan preferensi pada ruang publik tepi laut};
\draw [arrow] (fit) -- (soc);
\draw [arrow] (soc) -- (rrt);

\draw [arrow] (rrt) -- (met);

\draw [arrow] (met) -- (elm);


\draw [arrow] (elm) -- (rec);
\draw [arrow] (rec) -- (jun);
\draw [arrow] (jun) -- (tem);

\end{tikzpicture}
\caption{Kerangka Konseptual}
\label{fig:conc}
\end{figure}



%%
\begin{longtable}{ p{.33\textwidth}  p{.12\textwidth} p{.12\textwidth}  p{.12\textwidth}  p{.12\textwidth} }
	\caption{Proposisi Penelitian} \\
	\label{tab:props} \\
\toprule
  \textbf{Elemen RTP tepi laut} &
 {\rotatebox[origin=lB]{90}{\parbox[t]{2.6cm}{\textbf{Aksesibilitas}}}} & {\rotatebox[origin=lB]{90}{\parbox[t]{2.6cm}{\textbf{Estetika}}}} & {\rotatebox[origin=lB]{90}{\parbox[t]{2.6cm}{\textbf{Keamanan}}}} & {\rotatebox[origin=lB]{90}{\parbox[t]{2.6cm}{\textbf{Fasilitas}}}} \\
 \midrule
\textit{Jumlah pohon}  &  & \checkmark  & & \\
\tabitems Sedikit pohon  &  &  & & \\
\tabitems Beberapa pohon  &  &  & & \\
\textit{Bentuk pohon}  &  &  & \checkmark  & \checkmark \\
\tabitems Cukup rindang  & & &  & \\
\tabitems Rindang  & & &  & \\
\textit{Lebar jalan}  & \checkmark &  & & \checkmark \\
\tabitems 1-3m  &  & & & \\
\tabitems $\leq$ 3m &  & & & \\

\textit{Permukaan jalan}  & \checkmark  &  & & \\
\tabitems Paving  &  &  & &  \\
\tabitems Aspal  & &  & &  \\
\tabitems Tanah  &  & & &  \\

\textit{Warna bunga/tanaman}  &  & \checkmark & & \\
\tabitems satu atau dua warna   & &  & & \\
\tabitems tiga atau lebih warna  & &  & & \\
\textit{Jenis kursi}  &  &  & & \checkmark \\
\tabitems Kursi bergerak  & & & &  \\
\tabitems Kursi dinding  & & & &  \\

\textit{Pencahayaan jalan}  &  &  & \checkmark & \\
\tabitems Sedang   & & & &  \\
\tabitems Tinggi  & & & &  \\

\textit{Orientasi Elemen}  &  &  &  & \checkmark\\
\tabitems Membelakangi laut  & & & &  \\
\tabitems Menghadap laut   & & & &  \\

\textit{Tempat Wisata air}  & \checkmark  &  &  & \checkmark\\
\tabitems Tempat memancing   & & & &  \\
\tabitems Tempat berenang   & & & &  \\

\textit{Bangunan penunjang}  &  &  &  & \checkmark\\
\tabitems Stan \textit{(booth container)}   & & & &  \\
\tabitems Kedai   & & & &  \\

\textit{Elemen air}  &  &  & \checkmark & \\
\tabitems Laut yang tenang   & & & &  \\
\tabitems Laut yang berombak   & & & &  \\

% &&\\
% Umur &  & \multicolumn{2}{c}{Mempengaruhi}   \\
% Jenis kelamin &  & \multicolumn{2}{c}{Mempengaruhi}   \\
% Suku &  & \multicolumn{2}{c}{Mempengaruhi}   \\
% Pekerjaan &  & \multicolumn{2}{c}{Mempengaruhi}   \\

 \bottomrule
%\multicolumn{5}{l}{\footnotesize{\textit{* \qquad \checkmark : berhubungan}}}       \\

\end{longtable}



\begin{figure}[htbp]
\centering
\begin{tikzpicture}[node distance=2cm]

    \node (waf) [startstop, text width= 3cm] {Waterfront};

    \node (man) [startstop, left of=waf, xshift=-3cm, yshift=-1cm, text width= 4.5cm] {Manfaat Waterfront\\ 1. Sosial \\ 2. Rekreasi \\ 3. Transportasi \\ 4. Penggunaan Industri \\ 5. Persimpangan };

    \node (keb) [startstop, below of=waf, text width= 3cm] {Kebutuhan Masyarakat};

    \node (rup) [startstop, below of=keb, xshift=-3cm, text width= 3cm] {Ruang publik};

    \node (att) [startstop, below of=rup, text width= 4cm] {\textbf{Atribut Ruang}\\\small(aksesibilitas,fasilitas,estetika,keamanan, dan pemeliharaan)};

    \node (jun) [startstop, below of=keb, xshift=3cm, text width= 3cm] {Pengunjung};

    \node (kar) [startstop, below of=jun, text width= 5cm] {\textbf{Karakter Pengunjung}\\\small(umur,gender,ras dan status kegiatan ekonomi)};

    \node (pre) [startstop, below of=att, xshift=3cm, yshift=-1cm, text width= 4cm] {\bfseries Preferensi pengunjung terhadap ruang publik};

\draw [arrow] (waf) -- (man);
\draw [arrow] (waf) -- (keb);
\draw [latex'-latex'] (man) -- (keb);
\draw [arrow] (keb) -- (rup);
\draw [arrow] (rup) -- (att);
\draw [arrow] (keb) -- (jun);
\draw [arrow] (jun) -- (kar);

\draw [latex'-latex'] (att) -- (kar);

\draw [arrow] (att) -- (pre);
\draw [arrow] (kar) -- (pre);
\end{tikzpicture}
\caption{Kerangka Konseptual}
\end{figure}

%%%----------------------------------------------------------------------------------------------------------------------------------------------------
\begin{figure}[htbp]
\centering
\begin{tikzpicture}[node distance=2cm]

	\node (tit) [startstop, text width= 5cm] {Fitur Fisik Binaan pada Aktivitas Luar Jl. Pinggir Laut};
	\node (va1) [startstop, below of=tit, text width=5cm, xshift=-3cm] {Variabel Bebas\\ Fitur Fisik Binaan};
	\node (va2) [startstop, below of=tit, text width=5cm, xshift=3cm] {Variabel Tergantung\\ Aktivitas Luar};
	\node (de1) [startstop, below of=va1, text width=5cm, yshift=-2cm] {
		\textbf{Sub Variabel Bebas}\\
		- Elemen Jalan \\
		- Kualitas Jalan \\
		- Elemen Tempat Duduk \\
		- Kualitas Tempat Duduk \\
		- Elemen Alami \\
		- Kualitas Alami \\
		- Fasilitas \& Aminities \\
		- Estetika \\
	};
	\node (de2) [startstop, below of=va2, text width=5cm, yshift=-2cm] {
			\textbf{Sub Variable Tergantung}\\
		- Aktivitas relaxsasi\\
		- Aktivitas fisik\\
		- Travel aktif\\
		- Interaction with wildlife and nature\\
		- Interaksi sosial\\
		- Partisipasi di aktivitas grup\\
		};

\draw [arrow] (tit) -| (va1);
\draw [arrow] (va1) -- (de1);
\draw [arrow] (tit) -| (va2);
\draw [arrow] (va2) -- (de2);

\end{tikzpicture}
\caption{Flowchart sample 2}
\end{figure}

\end{comment}


%\onlyinsubfile{\biblio}
\end{document}
