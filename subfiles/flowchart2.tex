\documentclass[../main.tex]{subfiles}

\begin{document}

\begin{figure}[htbp]
\centering
\begin{tikzpicture}[node distance=2cm]

	\node (fit) [startstop, text width= 6cm] {\textbf{Elemen Ruang Publik}\\\small(Aksesibilitas, Estetika, Keamanan dan Fasilitas)};

	\node (pre) [startstop, right of=fit, text width= 5cm, xshift=4cm] {\textbf{Preferensi Pengunjung}\\ Kognisi, Afektif, Interpretatif, Evaluatif };

	\node (soc) [startstop, below of=fit, text width= 8cm, xshift= 3cm, yshift=-1cm ] {\textbf{Preferensi Ruang}};



\draw [arrow] (fit) -- (soc);
\draw [arrow] (pre) -- (soc);


\end{tikzpicture}
\caption{Kerangka Konseptual}
\end{figure}


\begin{figure}[htbp]
\centering
\begin{tikzpicture}[node distance=2cm]

    \node (waf) [startstop, text width= 3cm] {Waterfront};

    \node (man) [startstop, left of=waf, xshift=-3cm, yshift=-1cm, text width= 4.5cm] {Manfaat Waterfront\\ 1. Sosial \\ 2. Rekreasi \\ 3. Transportasi \\ 4. Penggunaan Industri \\ 5. Persimpangan };

    \node (keb) [startstop, below of=waf, text width= 3cm] {Kebutuhan Masyarakat};

    \node (rup) [startstop, below of=keb, xshift=-3cm, text width= 3cm] {Ruang publik};

    \node (att) [startstop, below of=rup, text width= 4cm] {\textbf{Atribut Ruang}\\\small(aksesibilitas,fasilitas,estetika,keamanan, dan pemeliharaan)};

    \node (jun) [startstop, below of=keb, xshift=3cm, text width= 3cm] {Pengunjung};

    \node (kar) [startstop, below of=jun, text width= 5cm] {\textbf{Karakter Pengunjung}\\\small(umur,gender,ras dan status kegiatan ekonomi)};

    \node (pre) [startstop, below of=att, xshift=3cm, yshift=-1cm, text width= 4cm] {\bfseries Preferensi pengunjung terhadap ruang publik};

\draw [arrow] (waf) -- (man);
\draw [arrow] (waf) -- (keb);
\draw [latex'-latex'] (man) -- (keb);
\draw [arrow] (keb) -- (rup);
\draw [arrow] (rup) -- (att);
\draw [arrow] (keb) -- (jun);
\draw [arrow] (jun) -- (kar);

\draw [latex'-latex'] (att) -- (kar);

\draw [arrow] (att) -- (pre);
\draw [arrow] (kar) -- (pre);
\end{tikzpicture}
\caption{Kerangka Konseptual}
\end{figure}

%%%----------------------------------------------------------------------------------------------------------------------------------------------------
\begin{figure}[htbp]
\centering
\begin{tikzpicture}[node distance=2cm]

	\node (tit) [startstop, text width= 5cm] {Fitur Fisik Binaan pada Aktivitas Luar Jl. Pinggir Laut};
	\node (va1) [startstop, below of=tit, text width=5cm, xshift=-3cm] {Variabel Bebas\\ Fitur Fisik Binaan};
	\node (va2) [startstop, below of=tit, text width=5cm, xshift=3cm] {Variabel Tergantung\\ Aktivitas Luar};
	\node (de1) [startstop, below of=va1, text width=5cm, yshift=-2cm] {
		\textbf{Sub Variabel Bebas}\\
		- Elemen Jalan \\
		- Kualitas Jalan \\
		- Elemen Tempat Duduk \\
		- Kualitas Tempat Duduk \\
		- Elemen Alami \\
		- Kualitas Alami \\
		- Fasilitas \& Aminities \\
		- Estetika \\
	};
	\node (de2) [startstop, below of=va2, text width=5cm, yshift=-2cm] {
			\textbf{Sub Variable Tergantung}\\
		- Aktivitas relaxsasi\\
		- Aktivitas fisik\\
		- Travel aktif\\
		- Interaction with wildlife and nature\\
		- Interaksi sosial\\
		- Partisipasi di aktivitas grup\\
		};

\draw [arrow] (tit) -| (va1);
\draw [arrow] (va1) -- (de1);
\draw [arrow] (tit) -| (va2);
\draw [arrow] (va2) -- (de2);

\end{tikzpicture}
\caption{Flowchart sample 2}
\end{figure}


%\onlyinsubfile{\biblio}
\end{document}
