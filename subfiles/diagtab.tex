\documentclass[../graphandtab.tex]{subfiles}


\begin{document}

\begin{table}
	\caption{Cross tabulasi ruang*aspek}
	\label{tab:ctpaE}
\begin{tabular}[t]{r P{4.3em}
    P{4.3em}P{4.3em}P{4.8em}}
\hline
\bfseries\diagbox[innerleftsep=10pt,innerrightsep=3pt,width=7em, height=2cm]{Ruang}{Aspek\\Ruang}&
 {\rotatebox[origin=c]{90}{\parbox[c]{2cm}{\textbf{Aksesibilitas}}}} & {\rotatebox[origin=c]{90}{\parbox[c]{2cm}{\textbf{Keamanan}}}} & {\rotatebox[origin=c]{90}{\parbox[c]{2cm}{\textbf{Estetika}}}} & {\rotatebox[origin=c]{90}{\parbox[c]{2cm}{\textbf{Fasilitas}}}} \\
 \toprule
Ruang A  & 32 (38\%) & 5 (38\%)   & 37 (44\%) & 30 (35\%) \\
Ruang B  & 15 (18\%) & 2 (8\%)   & 9 (11\%) & 20 (24\%) \\
Total  & 47 (45\%) & 7 (55\%)  & 46 (54\%) & 50 (59\%) \\

 \bottomrule
\multicolumn{5}{l}{\rule{0pt}{1em}\scriptsize Sumber: analisis, 2022 }\\
\end{tabular}
\end{table}

\end{document}
