\documentclass[../main.tex]{subfiles}

\begin{document}

\section{FlowChart Sample 1}
This environment should uncomment package of TiKz and also configuration of TiKz flowchart in .cls file.

\begin{figure}[hp]
\centering
\begin{tikzpicture}[node distance=2cm]

\node (ltr) [startstop] {Latar Belakang};
\node (rum) [startstop, right of=ltr, xshift=2cm] {Perumusan Masalah};
\node (tuj) [startstop, below of=rum, yshift=0.5cm] {Tujuan Penelitian};
\node (pus) [startstop, below of=tuj, yshift=0.5cm] {Studi Pustaka};
\node (kaj) [startstop, below of=pus, text width=3.5cm, xshift= -4cm, yshift=.5cm] {
	\textbf{Kajian Teori}\\ - Fitur binaan\\ - Aktivitas Luar
};
\node (kaj2) [startstop, below of=pus, text width=3.5cm, xshift= 4cm, yshift=.5cm] {
	\textbf{Gambaran Objek}\\ Fitur Binaan dan Aktivitas Luar Jl. Pinggir Laut
};
\node (hip) [startstop, below of=pus, yshift=-.5cm] {Hipotesa};
\node (met) [startstop, below of=hip, yshift=-.75cm, text width=7cm] {
	\textbf{Metode Peneltian}\\ Menggunakan Metode penelitian Kuantitatif Rasionalistik

	\textbf{Variabel}\\
	- Bebas : Fitur Binaan\\
	- Terikat : Aktivitas Luar\\

	\textbf{Sumber data}: Observasi dan Kuesioner
};
\node (ana) [startstop, below of=met, text width=8cm, yshift=-2cm] {
		\textbf{Analisis Data Statistik}\\ Penelitian ini menggunakan metode statika berupa uji regresi guna mengetahui pengaruh variabel fitur binaan terhadap variabel aktivitas luar.
};
\node (tem) [startstop, below of=ana, yshift=-.25cm] {Temuan Penelitian};
\node (kes) [startstop, below of=tem, yshift=.6cm] {Kesimpulan dan Rekomendasi};

\draw [arrow] (ltr) -- (rum);
\draw [arrow] (rum) -- (tuj);
\draw [arrow] (tuj) -- (pus);
\draw [arrow] (pus) -| (kaj);
\draw [arrow] (pus) -| (kaj2);
\draw [doublearrow] (kaj) -- (kaj2);
\draw [arrow] (kaj) |- (met);
\draw [dotted] (kaj) |- (hip);
\draw [arrow] (kaj2) |- (met);
\draw [dotted] (kaj2) |- (hip);
\draw [arrow] (met) -- (ana);
\draw [arrow] (ana) -- (tem);
\draw [arrow] (tem) -- (kes);

\end{tikzpicture}
\caption{Flowchart sample 1}
\end{figure}



%\onlyinsubfile{\biblio}
\end{document}
