\documentclass[../main.tex]{subfiles}

\begin{document}


\begin{figure}[hp]
\centering
\begin{tikzpicture}[node distance=2cm]


\node (man) [startstop, text width=6cm] {\textbf{Manusia} \\ 1. Harapan \quad 5 Pendidikan\\ 2. Pengalaman \quad 6. Budaya\\ 3.Konteks sosial \quad 7. Motivasi \\ 4. Informasi \quad 8. Personaliti};

\node (lan) [startstop, right of=man, text width=6cm, xshift=4.5cm] {
	\textbf{Lanskap}\\ 1. Elemen fisik  2. Kontruksi komposisi  3. Konteks lokasi  4. Kealamian  5. Fitur buatan  6. Suara  7. Bau  8. Orang  9. Gestalt};

\node (int) [startstop, below of=man, text width=12cm, xshift=3cm, yshift=-2cm] {
		\textbf{Interaksi} \\ 1. Orang-orang-lanskap 2. orang-kelompok-lanskap 3. orang-lanskap 4. aktif 5. pasif 6. Sengaja 7. Kebetulan 8. Unik 9. Biasa

         };

\node (out) [startstop, below of=int, text width = 12cm, yshift=-2cm] {\textbf{Akibat} \\ 1. Informasi 2. Kepuasan 3. Kesejahteraan 4. Aktifitas fisik 5. Stimulasi 6. Tempat berlindung 7. Kesempatan 8. Nilai 9. Persamaan prediktif 10. Elemen lanskap yang menonjol 11. Perilaku kebiasaan 12. Ketakutan};


\draw [doublearrow] (man) -- (int);

\draw [doublearrow] (lan) -- (int);

\draw [doublearrow] (int) -- (out);

\end{tikzpicture}
\caption{Kerangka Pemikiran}
\label{fig:pik}
\end{figure}


%\onlyinsubfile{\biblio}
\end{document}
