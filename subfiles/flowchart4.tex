\documentclass[../main.tex]{subfiles}

\begin{document}
\begin{comment}


\begin{figure}[hp]
\centering
\begin{tikzpicture}[node distance=2cm]


\node (man) [startstop, text width=6cm] {\textbf{Manusia} \\ 1. Harapan \quad 5 Pendidikan\\ 2. Pengalaman \quad 6. Budaya\\ 3.Konteks sosial \quad 7. Motivasi \\ 4. Informasi \quad 8. Personaliti};

\node (lan) [startstop, right of=man, text width=6cm, xshift=4.5cm] {
	\textbf{Lanskap}\\ 1. Elemen fisik  2. Kontruksi komposisi  3. Konteks lokasi  4. Kealamian  5. Fitur buatan  6. Suara  7. Bau  8. Orang  9. Gestalt};

\node (int) [startstop, below of=man, text width=12cm, xshift=3cm, yshift=-2cm] {
		\textbf{Interaksi} \\ 1. Orang-orang-lanskap 2. orang-kelompok-lanskap 3. orang-lanskap 4. aktif 5. pasif 6. Sengaja 7. Kebetulan 8. Unik 9. Biasa

         };

\node (out) [startstop, below of=int, text width = 12cm, yshift=-2cm] {\textbf{Akibat} \\ 1. Informasi 2. Kepuasan 3. Kesejahteraan 4. Aktifitas fisik 5. Stimulasi 6. Tempat berlindung 7. Kesempatan 8. Nilai 9. Persamaan prediktif 10. Elemen lanskap yang menonjol 11. Perilaku kebiasaan 12. Ketakutan};


\draw [doublearrow] (man) -- (int);

\draw [doublearrow] (lan) -- (int);

\draw [doublearrow] (int) -- (out);

\end{tikzpicture}
\caption{Kerangka Pemikiran}
\label{fig:pik}
\end{figure}

\end{comment}

\begin{figure}[htbp]
\small
\centering
\begin{tikzpicture}[node distance=1cm]

    \node (fit) [startstop, text width=.89\textwidth] {\textbf{Permalahan penelitian}\\ 1. Apa saja ruang yang dicenderungi pengunjung terhadap ruang publik?  \\ 2. Mengapa orang memiliki preferensi pada ruang tersebut? \\ 3. Apa saja elemen-elemen pada suatu ruang sehingga orang memilih ruang tersebut? };

	\node (soc) [startstop, below of=fit, text width=.89\textwidth, yshift=-2cm] {\textbf{Tujuan penelitian}\\
    1.  Untuk menganalisis ruang yang cenderung menjadi preferensi pengunjung terhadap ruang A dan B.\\
    2.  Untuk menganalisis mengapa orang memiliki preferensi ruang tersebut.\\
    3.  Untuk menganalisis elemen-elemen apa pada ruang yang menjadi preferensi.\\

        };

	\node (met) [startstop, below of=soc, text width=.89\textwidth, yshift=-1.5cm] {\textbf{Metode penelitian}\\ Pendekatan penilitian metode campuran dengan  menggunakan kuesioner survei \textit{mixed-methodology}};

    \node (pro) [startstop, below of=met, text width=.89\textwidth, yshift=-1cm] {\textbf{Proposisi}\\ Penyusunan proposisi penelitian berdasarkan tinjauan teori terkait yang disesuaikan dengan objek dan tujuan penelitian untuk dijadikan pedoman dalam pengumpulan dan analisis data.
    };

    \node (ker) [startstop, below of=pro,  yshift=-.8cm, text width=6cm] {\textbf{Kerangka indikator penelitian}    };

    \node (ask) [startstop, below of=ker, xshift=-2.2cm, yshift=-.8cm, text width= 6.2cm] {\textbf{ASPEK RUANG}\\

\begin{tabular}{p{.4\textwidth}@{}p{.4\textwidth}}
    \tabitem Aksesibilitas & \tabitem Keamanan \\
    \tabitem Estetika & \tabitem Fasilitas\\
\end{tabular}};

    \node (elm) [startstop, below of=ask, yshift= -3cm, text width= 6.2cm] {\textbf{ELEMEN RUANG}

\begin{tabular}{p{.4\textwidth}@{}p{.4\textwidth}}
    \tabitem Pohon yang banyak & \tabitem Jenis kursi \\
    \tabitem Pohon yang cukup rindang & \tabitem Bunga yang berwarna\\
    \tabitem Pohon yang sangat rindang & \tabitem Jalan yang lebar\\
    \tabitem Tatanan dimana orang agak jauh & \tabitem Kehadiran elemen \\
\end{tabular}};


    \node (jun) [startstop, below of=ker, xshift=4cm, yshift= -2.4cm, text width= 5cm] {Menghasilkan temuan preferensi pada ruang publik tepi laut};

\draw [arrow] (fit) -- (soc);
\draw [arrow] (soc) -- (met);
\draw [arrow] (met) -- (pro);
\draw [arrow] (pro) -- (ker);

\draw [doublearrow] (ask) -- (elm);

\draw [arrow] (ask) -| (jun);
\draw [arrow] (elm) -| (jun);

\end{tikzpicture}
\caption{Kerangka Konseptual}
\label{fig:conc}
\end{figure}


%\onlyinsubfile{\biblio}
\end{document}
